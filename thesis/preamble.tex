


%%%
%Mathematik
%
\usepackage[]{amsmath} % Viele Mathematik-Sachen: Doku: /usr/share/doc/texmf/latex/amsmath/amsldoc.dvi.gz
\PassOptionsToPackage{fleqn,leqno}{amsmath} % options must be passed this way, otherwise it does not work with glossaries
%fleqn (=Gleichungen linksbündig platzieren) funktioniert nicht direkt. Es muss noch ein Patch gemacht werden:
%\addtolength\mathindent{1em}%work-around ams-math problem with align and 9 -> 10. Does not work with glossaries, No visual changes.
\usepackage{mathtools} %fixes bugs in AMS math
%\usepackage{amsfonts}
\usepackage{amssymb}
%
%for theorems, replacement for amsthm
\usepackage[amsmath,hyperref]{ntheorem}
\theorempreskipamount 2ex plus1ex minus0.5ex
\theorempostskipamount 2ex plus1ex minus0.5ex
\theoremstyle{break}
\theoremheaderfont{\itshape} 
\theorembodyfont{\upshape}
\newtheorem{definition}{Definition}[section]


%
%%%




%%%
% Fonts and typesetting 
\usepackage{mathspec}
%\setmainfont[BoldFont={Source S Pro Semibold}, ItalicFont={Source Sans Pro Italic}, BoldItalicFont={Source Sans Pro Semibold Italic}]{Source Sans Pro}
\setprimaryfont[
	UprightFont={* Light},
	BoldFont={* Semibold},
	ItalicFont={* Light Italic},     %Requires additional installation of self-build OTFs, because not published yet. 
	BoldItalicFont={* Semibold Italic} 
	] 
	{Source Serif Pro}
\setallsansfonts[
	UprightFont={* Light},
	BoldFont={* Semibold},
	ItalicFont={* Light Italic},
	BoldItalicFont={* Semibold Italic},
	Scale=MatchLowercase]
	{Source Sans Pro}%
%\setmainfont[BoldFont={Source Sans Pro Semibold}, ItalicFont={Source Sans Pro Italic}, BoldItalicFont={Source Sans Pro Semibold Italic}]{Source Sans Pro}[Scale=MatchLowercase]
\setallmonofonts[
	UprightFont={* Light},
	BoldFont={* Semibold},
	Scale=MatchLowercase]
	{Source Code Pro}%

\setmathfont(Greek)[
	UprightFont={* Light},
	BoldFont={* Semibold},
	AutoFakeSlant=0.15
	]
{Source Serif Pro}	 


\newfontfamily\dejavu{DejaVu Sans}[
	UprightFont={*},
	BoldFont={* Bold},
	ItalicFont={* Oblique},
	BoldItalicFont={* Bold Oblique},
]

% Fixing Kerning for Hat and Bar in Math mode
\newcommand*{\xhat}[1]{#1\kern-0.33em\hat{\phantom{#1}}}
\newcommand*{\xxhat}[1]{#1\kern-0.30em\hat{\phantom{#1}}}

\newcommand*{\xbar}[1]{#1\kern-0.33em\bar{\phantom{#1}}}

%\normalizevarforms 
	        
% Fix mathspec before amsmath bug
% https://tex.stackexchange.com/questions/85696/what-causes-this-strange-interaction-between-glossaries-and-amsmath

\makeatletter % undo the wrong changes made by mathspec
\let\RequirePackage\original@RequirePackage
\let\usepackage\RequirePackage
\makeatother


%
%%%





%%%
% Memoir Class Page formatting
%
% Change line witdh to new font
\setlxvchars \setxlvchars 
% 5mm binding tollerance
\setbinding{5mm} 
\settypeblocksize{600pt}{1.1\lxvchars}{*}
\setlrmargins{*}{*}{1.5}
\setulmargins{4cm}{*}{*}
\setheadfoot{2\onelineskip}{2\onelineskip}
\setmarginnotes{17pt}{3.5cm}{\onelineskip}
% Save changes
\checkandfixthelayout
% Headings and Page Style
\pagestyle{ruled}
\copypagestyle{chapter}{plain}
\makeevenfoot{chapter}{\thepage}{}{}
\makeoddfoot{chapter}{}{}{\thepage}
%
%%%


%%%
% Ortography
\usepackage[ngerman,english]{babel}
%
%%%



%%%
% Itemize, Enumerate, Lists
\firmlists
%
%%%



%%%
%Hyperref
\usepackage{hyperref}
\hypersetup{
	pdftitle    = {Privacy-Preserving Collaborative Filtering with SPDZ},
	pdfsubject  = {Master Thesis},
	pdfauthor   = {Thibaud René Kehler, 321038},
	pdfkeywords = {Privacy} {Recommender Systems} {Collaborative Filtering} {Secure Multiparty Computation} {SPDZ},
	pdfcreator  = {XeLaTex},
	hidelinks,
}
\urlstyle{sf}
%
%%%




%%%
% Superscripts for 1st, 2nd, 3rd, 4th
\usepackage[super]{nth}
%
%%%





%%%
% Colors
\usepackage{xcolor}
% Define RWTH style colors

%%%%%%%%%%%%%%%%%%%%%%%%%%%%%%%%%%%%%%%%%%
% For HKS colors use the spotcolor package
% which predefines a number of such colors
% Main blue color is HKS 44 K
%%%%%%%%%%%%%%%%%%%%%%%%%%%%%%%%%%%%%%%%%%

%%%%%%%%%%%%%%%%%%%%%%%%%%%%%%%%%%%%%%%%%%
% RGB colors used for screen
%%%%%%%%%%%%%%%%%%%%%%%%%%%%%%%%%%%%%%%%%%
% The main blue color, 100 and 50 are the
% most commonly used ones
% e.g. in the logo
%\definecolor{rwth}   {RGB}{  0  84 159}
%\definecolor{rwth-75}{RGB}{ 64 127 183}
%\definecolor{rwth-50}{RGB}{142 186 229}
%\definecolor{rwth-25}{RGB}{199 221 242}
%\definecolor{rwth-10}{RGB}{232 241 250}

% My colors
\definecolor{rwth}   {RGB}{  7  72 158}
\definecolor{rwth-75}{RGB}{ 58 112 183}
\definecolor{rwth-50}{RGB}{114 159 207}
\definecolor{rwth-25}{RGB}{190 211 233}
\definecolor{rwth-10}{RGB}{231 239 247}


% All the other colors
% Secondary colors
\definecolor{black}   {RGB}{  0   0   0}
\definecolor{black-75}{RGB}{100 101 103}
\definecolor{black-50}{RGB}{156 158 159}
\definecolor{black-25}{RGB}{207 209 210}
\definecolor{black-10}{RGB}{236 237 237}

\definecolor{magenta}   {RGB}{227   0 102}
\definecolor{magenta-75}{RGB}{233  96 136}
\definecolor{magenta-50}{RGB}{241 158 177}
\definecolor{magenta-25}{RGB}{249 210 218}
\definecolor{magenta-10}{RGB}{253 238 240}

\definecolor{yellow}   {RGB}{255 237   0}
\definecolor{yellow-75}{RGB}{255 240  85}
\definecolor{yellow-50}{RGB}{255 245 155}
\definecolor{yellow-25}{RGB}{255 250 209}
\definecolor{yellow-10}{RGB}{255 253 238}

% The extended color spectrum
\definecolor{petrol}   {RGB}{  0  97 101}
\definecolor{petrol-75}{RGB}{ 45 127 131}
\definecolor{petrol-50}{RGB}{125 164 167}
\definecolor{petrol-25}{RGB}{191 208 209}
\definecolor{petrol-10}{RGB}{230 236 236}

\definecolor{turkis}   {RGB}{  0 152 161}
\definecolor{turkis-75}{RGB}{  0 177 183}
\definecolor{turkis-50}{RGB}{137 204 207}
\definecolor{turkis-25}{RGB}{202 231 231}
\definecolor{turkis-10}{RGB}{235 246 246}

\definecolor{grun}   {RGB}{ 87 171  39}
\definecolor{grun-75}{RGB}{141 192  96}
\definecolor{grun-50}{RGB}{184 214 152}
\definecolor{grun-25}{RGB}{221 235 206}
\definecolor{grun-10}{RGB}{242 247 236}

\definecolor{maigrun}   {RGB}{189 205   0}
\definecolor{maigrun-75}{RGB}{208 217  92}
\definecolor{maigrun-50}{RGB}{224 230 154}
\definecolor{maigrun-25}{RGB}{240 243 208}
\definecolor{maigrun-10}{RGB}{249 250 237}

\definecolor{orange}   {RGB}{246 168   0}
\definecolor{orange-75}{RGB}{250 190  80}
\definecolor{orange-50}{RGB}{253 212 143}
\definecolor{orange-25}{RGB}{254 234 201}
\definecolor{orange-10}{RGB}{255 247 234}

\definecolor{rot}   {RGB}{204   7  30}
\definecolor{rot-75}{RGB}{216  92  65}
\definecolor{rot-50}{RGB}{230 150 121}
\definecolor{rot-25}{RGB}{243 205 187}
\definecolor{rot-10}{RGB}{250 235 227}

\definecolor{bordeaux}   {RGB}{161  16  53}
\definecolor{bordeaux-75}{RGB}{182  82  86}
\definecolor{bordeaux-50}{RGB}{205 139 135}
\definecolor{bordeaux-25}{RGB}{229 197 192}
\definecolor{bordeaux-10}{RGB}{245 232 229}

\definecolor{violett}   {RGB}{ 97  33  88}
\definecolor{violett-75}{RGB}{131  78 117}
\definecolor{violett-50}{RGB}{168 133 158}
\definecolor{violett-25}{RGB}{210 192 205}
\definecolor{violett-10}{RGB}{237 229 234}

\definecolor{lila}   {RGB}{122 111 172}
\definecolor{lila-75}{RGB}{155 145 193}
\definecolor{lila-50}{RGB}{188 181 215}
\definecolor{lila-25}{RGB}{222 218 235}
\definecolor{lila-10}{RGB}{242 240 247}

%
%%%




%%%
% Division Styles
\colorlet{chaptercolor}{rwth-50}
% helper macros
\newcommand\numlifter[1]{\raisebox{-2cm}[0pt][0pt]{\smash{#1}}}
\newcommand\numindent{\kern37pt}
\newlength\chaptertitleboxheight
\makechapterstyle{hansen}{
	\renewcommand\printchaptername{\raggedleft}
	\renewcommand\printchapternum{%
		\begingroup%
		\leavevmode%
		\chapnumfont%
		\strut%
		\numlifter{\thechapter}%
		\numindent%
		\endgroup%
	}
	\renewcommand
	*
	{\printchapternonum}{%
		\vphantom{\begingroup%
			\leavevmode%
			\chapnumfont%
			\numlifter{\vphantom{9}}%
			\numindent%
			\endgroup}
		\afterchapternum}
	\setlength\midchapskip{0pt}
	\setlength\beforechapskip{0.5\baselineskip}
	\setlength{\afterchapskip}{3\baselineskip}
	\renewcommand\chapnumfont{%
		\fontsize{4cm}{0cm}%
		\bfseries%
		\sffamily%
		\color{chaptercolor}%
	}
	\renewcommand\chaptitlefont{%
		\normalfont%
		\huge%
		\bfseries%
		\raggedleft%
	}%
	\settototalheight\chaptertitleboxheight{%
		\parbox{\textwidth}{\chaptitlefont \strut bg\\bg\strut}}
	\renewcommand\printchaptertitle[1]{%
		\parbox[t][\chaptertitleboxheight][t]{\textwidth}{%
			%\microtypesetup{protrusion=false}% add this if you use microtype
			\chaptitlefont\strut ##1\strut}%
	}
}
% Chapter Style
\chapterstyle{hansen}
% Section Style
\setsecheadstyle{\Large\bfseries\color{rwth-75}\raggedright}
% Subsection Style
\setsubsecheadstyle{\large\itshape\raggedright}

% Numbering
\setsecnumdepth{subsection}
\settocdepth{subsection}
% Write "Section 1.2" instead of "§1.2"
\renewcommand{\sectionrefname}{Section}
\renewcommand{\Sref}[1]{\sectionrefname~\ref{#1}}

% Command to remove a section from TOC
\newcommand{\nocontentsline}[3]{}
\newcommand{\tocless}[2]{{\let\addcontentsline=\nocontentsline#1{#2}}\noindent}
% e.g. \tocless\section{Title}
%
%%%



%%%
% Table of Content Fonts
%\renewcommand{\cftchapterfont}{\normalsize\normalfont}
%\renewcommand{\cftchapterformatpnum}{\normalsize\normalfont}
%
%%%




%%%
% Bibliography
\usepackage[natbib=true,%
maxcitenames=2,%
maxbibnames=9,%
sortlocale=auto,%
style=numeric-comp,%
backend=biber,%
doi=false,%
isbn=false,%
url=false,%
eprint=true,%
urldate=comp,%
giveninits=true%
]{biblatex}
\addbibresource{../literatur/literatur.bib}
%
%%%






%%%
% Acronyms & Glossaries
\usepackage[acronym,nomain,numberedsection]{glossaries}
\usepackage{glossary-longragged} 
\newglossary*{attribution}{Attributions}
\newacronym[\glslongpluralkey={privacy-enhancing technologies}]{pet}{PET}{privacy enhancing technology}
\newacronym{smpc}{SMPC}{secure multiparty computation}
\newacronym{mac}{MAC}{message authentication code}
%\newacronym{cf}{CF}{collaborative filtering}
\newacronym{is}{IS}{information science}
\newacronym{cs}{CS}{computer science}
\newacronym{cia}{CIA}{confidentiality, integrity and availability}
\newacronym{ot}{OT}{oblivious transfer}
\newacronym{he}{HE}{homomorphic encryption}
\newacronym{fhe}{FHE}{fully homomorphic encryption}
\newacronym{sfhe}{SHE}{somewhat fully homomorphic encryption}
\newacronym{phe}{FHE}{partially homomorphic encryption}
\newacronym[\glslongpluralkey={items of interest}]{ioi}{IoI}{item of interest}
\newacronym{dfd}{DFD}{data flow diagram}
\newacronym{cfip}{CFIP}{concern for information privacy}
\newacronym{iuipc}{IUIPC}{Internet user's information privacy concerns}
\newacronym{apco}{APCO}{antecedents, privacy concerns, outcomes}
\newacronym{linddun}{LINDDUN}{linkability, identifyability, non-repudiation, detectability, disclosure, unawareness, non-compliance }
\newacronym{svd}{SVD}{singular value decomposition}
\newacronym{prng}{PRNG}{pseudo-random number generator}
\newacronym{risc}{RISC}{reduced instruction set computer}
\newacronym{cisc}{CISC}{complex instruction set computer}
\newacronym{csr}{CSR}{compressed sparse row}
\newacronym{csc}{CSC}{compressed sparse collumn}
\newacronym{lil}{LIL}{list of lists}
\newacronym{mae}{MAE}{mean absolute error}
\newacronym{rmse}{RMSE}{root mean square error}
\newacronym{ram}{RAM}{random access memory}
\newacronym{oram}{ORAM}{oblivious random access memory}
\newacronym{api}{API}{application programming interface}
\newacronym{scapi}{SCAPI}{Secure Computation \glsname{api}}

\glsaddall[types=acronym] % Abkürzungen auch anzeigen, wenn sie nicht im Text vorkommen.
\newglossaryentry{attr1}{type={attribution},name={Urheberrechtlich Geschütztes-Logo},description={Vorname Nachname, single use permission}}
\makeglossaries
%
%%%





%%%
% Grafikeinbindungen
\usepackage{graphicx}
% Relocate default path
\graphicspath{{\getgraphicspath}}
\newcommand{\getgraphicspath}{../figures/}
%
%%%



%%%
% Tikz
\usepackage{tikz}
\usepackage{pgfplots}
\usepackage{pgfplotstable}
\pgfplotsset{compat=newest}
\usetikzlibrary{shapes, shapes.symbols, shapes.misc,
			 arrows, arrows.meta,
			 calc,
			 backgrounds,
			 positioning,
			 shadows
		 }
\tikzset{
	font={\sffamily\scriptsize},
	h1/.style={minimum height=9mm},
	h2/.style={minimum height=15mm},
	h3/.style={minimum height=30mm},
	w1/.style={minimum width=16mm, text width=16mm, inner xsep = 0mm},
	w2/.style={minimum width=32mm, text width=32mm, inner xsep = 0mm},
	w3/.style={minimum width=48mm, text width=48mm, inner xsep = 0mm},
	ww2/.style={minimum width=33mm, text width=33mm, inner xsep = 0mm},
	ww3/.style={minimum width=49mm, text width=49mm, inner xsep = 0mm},
	ww4/.style={minimum width=65mm, text width=65mm, inner xsep = 0mm},
	ww6/.style={minimum width=65mm, text width=97mm, inner xsep = 0mm},
	www3/.style={minimum width=49mm, text width=49mm, inner xsep = 0mm},
	www4/.style={minimum width=66mm, text width=66mm, inner xsep = 0mm},
	www5/.style={minimum width=83mm, text width=83mm, inner xsep = 0mm},
	www6/.style={minimum width=98mm, text width=98mm, inner xsep = 0mm},
	semirect/.style n args={4}{
		draw=none,
		rectangle,
		append after command={
			\pgfextra{%
				\pgfkeysgetvalue{/pgf/outer xsep}{\oxsep}
				\pgfkeysgetvalue{/pgf/outer ysep}{\oysep}
				\def\arg@one{#1}
				\def\arg@two{#2}
				\def\arg@three{#3}
				\def\arg@four{#4}
				\begin{pgfinterruptpath}
					\ifx\\#1\\\else
					\draw[draw,#1] ([xshift=-\oxsep,yshift=+\pgflinewidth]\tikzlastnode.south east) edge ([xshift=-\oxsep,yshift=0\ifx\arg@two\@empty-\pgflinewidth\fi]\tikzlastnode.north east);
					\fi\ifx\\#2\\\else
					\draw[draw,#2] ([xshift=-\pgflinewidth,yshift=-\oysep]\tikzlastnode.north east) edge ([xshift=0\ifx\arg@three\@empty+\pgflinewidth\fi,yshift=-\oysep]\tikzlastnode.north west);
					\fi\ifx\\#3\\\else
					\draw[draw,#3] ([xshift=\oxsep,yshift=0-\pgflinewidth]\tikzlastnode.north west) edge ([xshift=\oxsep,yshift=0\ifx\arg@four\@empty+\pgflinewidth\fi]\tikzlastnode.south west);
					\fi\ifx\\#4\\\else
					\draw[draw,#4] ([xshift=0+\pgflinewidth,yshift=\oysep]\tikzlastnode.south west) edge ([xshift=0\ifx\arg@one\@empty-\pgflinewidth\fi,yshift=\oysep]\tikzlastnode.south east);
					\fi
				\end{pgfinterruptpath}
			}
		}
	},
	every node/.style={align=center},
	entity/.style={rectangle, draw=black, semithick, w1,align=center, fill=white},
	process/.style={ellipse, draw=black, semithick, w1, h1,align=center, fill=white},
	datastore/.style={semirect={}{draw=black, semithick}{}{draw=black, semithick}, w1,align=center},
	dataflow/.style={-{latex}, thick},
	ddataflow/.style={{latex}-{latex}, thick},
	boundary/.style={dash pattern=on 5pt off 2pt, very thick,rounded corners},
	% Architecture Elements
	module/.style={rounded corners=10pt, draw=black, semithick, minimum height=20pt},
	component/.style={rounded corners=5pt, draw=black, semithick, minimum height=10pt},
	file/.style={draw, minimum height=3em, minimum width=2em, font=\ttfamily\scriptsize,
		fill=white, 
		double copy shadow={shadow xshift=3pt, 
			shadow yshift=3pt, fill=white, draw}
	},
	files/.style={draw, minimum height=4em, minimum width=3em, font=\ttfamily\scriptsize,
		fill=white, 
		double copy shadow={shadow xshift=3pt, 
			shadow yshift=3pt, fill=white, draw}
		},
	py/.style={fill=rwth-25, draw opacity=0, fill opacity=1},
	mpc/.style={fill=grun-25, draw opacity=0, fill opacity=1},
	mix/.style={fill=petrol-25, draw opacity=0, fill opacity=1},
	new/.style={draw opacity=1, draw=black, fill opacity=1},
	mod/.style={draw opacity=1, draw=black, fill opacity=1, dash pattern=on 5pt off 2pt},
	% Thick arrows
	thickarrow/.style={shorten >= 1mm,shorten <=1mm, draw=black-50, line width = 5mm},
	rthickarrow/.style={-{LaTeX[width=10mm,length=5mm]}, thickarrow},
	lthickarrow/.style={{LaTeX[width=10mm,length=5mm]}-, thickarrow},
	rlthickarrow/.style={{LaTeX[width=10mm,length=5mm]}-{LaTeX[width=10mm,length=5mm]}, thickarrow},
}

\pgfplotscreateplotcyclelist{rwth color}{%
	rwth-75,
	rot-75,
	grun-75,
	petrol-75,
	bordeaux-75,
	maigrun-75,
	turkis-75,
	lila-75,
	orange-75,
	violett-75,
	magenta-75,
	yellow-75,
	violett-75,
	orange-75
}


\pgfplotscreateplotcyclelist{rwth}{%
	rwth-75,every mark/.append style={fill=rwth},mark=*\\%
	rot!80!white,every mark/.append style={fill=rot},mark=square*\\%
	grun-75,every mark/.append style={fill=grunk},mark=otimes*\\%
	black,mark=star\\%
	rwth-75,every mark/.append style={fill=rwth},mark=diamond*\\%
	rot!80!white,densely dashed,every mark/.append style={solid,fill=rot},mark=*\\%
	grun-75,densely dashed,every mark/.append style={
		solid,fill=grun},mark=square*\\%
	black,densely dashed,every mark/.append style={solid,fill=black-75},mark=otimes*\\%
	rwth-75,densely dashed,mark=star,every mark/.append style=solid\\%
	rot-75,densely dashed,every mark/.append style={solid,fill=rot},mark=diamond*\\%
}

\pgfkeys{/pgf/number format/.cd,1000 sep={\,}}

%
%%%

%%%
% SI Einheiten
\usepackage[binary-units]{siunitx}
\sisetup{
detect-all, % select font from suroundings
range-phrase = --,
list-final-separator = {, },
group-digits = integer,
group-minimum-digits = 4
}
%
%%%


%%%
% Floats General
%
% Float Barrier
\setFloatBlockFor{section}
% Subcaption font
\subcaptionsize{\normalsize}
\subcaptionlabelfont{\normalfont}
\subcaptionfont{\normalfont}
% Spacing
\tightsubcaptions
\setlength{\floatsep}{10pt}
\setlength{\textfloatsep}{20pt}
\setlength{\intextsep}{10pt}

\captionstyle{}
\changecaptionwidth
\captionwidth{0.8\linewidth}
\captionnamefont{\bfseries}
%%%

%%%
% Figures
\newsubfloat{figure}
\setfloatadjustment{figure}{\centering}
%
%%%

%%%
% Table
\setfloatadjustment{table}{\centering}
%
%%%

%%%
% Listings
\usepackage{listings}

\lstset{language=Python,
	showstringspaces=false,
	extendedchars=true,
	basicstyle=\footnotesize\ttfamily,
	commentstyle=\slshape,
	stringstyle=\ttfamily, %Original: \rmfamily, damit werden die Strings im Quellcode hervorgehoben. Zusaetzlich evtl.: \scshape oder \rmfamily durch \ttfamily ersetzen. Dann sieht's aus, wie bei fancyvrb
	breaklines=true,
	breakatwhitespace=true,
	columns=flexible,
	aboveskip=\fboxsep, %deaktivieren, falls man lstlistings direkt als floating object benutzt (\begin{lstlisting}[float,...])
	belowskip=\fboxsep, %deaktivieren, falls man lstlistings direkt als floating object benutzt (\begin{lstlisting}[float,...])
	frame=single,
	framesep=3\fboxsep,
	framexbottommargin=0pt,
	xleftmargin=3\fboxsep,
	xrightmargin=3\fboxsep
}

\newcommand{\inlinelst}[2]{\colorbox{lightgray}{\lstinline[language=#1]$#2$}}
\newcommand{\code}{\texttt}

\newfloat[chapter]{fltlisting}{lst}{\lstlistingname}
\newcommand{\lref}[1]{\lstlistingname~\ref{#1}}
\newsubfloat{fltlisting}
%\setfloatadjustment{fltlisting}{\loosesubcaptions}
%
%%%


%%%
% Algorithms
\usepackage[noend]{algpseudocode}
\newcommand{\algorithmname}{Algorithm}
\newfloat[chapter]{algorithm}{alg}{\algorithmname}
\newcommand{\aref}[1]{\algorithmname~\ref{#1}}
%\algrenewcommand\textproc{\texttt}
%
%%%

%%%
% Protocols
\newcommand{\protocolname}{Protocol}
\newfloat[chapter]{protocol}{prot}{\protocolname}
\newcommand{\prref}[1]{\protocolname~\ref{#1}}
%
%%%


%%%%
% Tables
\DisemulatePackage{tabularx}
\usepackage{multirow,tabu,booktabs}
\tabulinesep = 0pt
%
%%%

%%%
% Counters
\letcountercounter{table}{figure}
\letcountercounter{protocol}{figure}
\letcountercounter{algorithm}{figure}
\letcountercounter{fltlisting}{figure}
%
%%%

%%%
% Narrow Framed Boxes
\newenvironment{nframed}{% % uses default \FrameCommand
	\setlength{\topsep}{0pt}%
	\MakeFramed{\advance\hsize -\width \FrameRestore}}%
{\endMakeFramed}
% Color leftbar
\renewenvironment{leftbar}{%
	\def\FrameCommand{{\color{rwth-75}\vrule width 3pt} \hspace{10pt}}%
	\MakeFramed{\advance\hsize -\width \FrameRestore}}%
{\endMakeFramed}
%
%%%


%%% 
% Quotes
% Zitate in \enquote{...} setzen, dann werden automatisch die richtigen Anführungszeichen verwendet.
\usepackage{csquotes}
%
%%%

%%% 
% Center enumerations
\usepackage{varwidth}
%
%%%



%%%
% Abbreviations
\usepackage{xspace}
\newcommand*{\eg}[0]{e.g.\@\xspace}
\newcommand*{\ie}[0]{i.e.\@\xspace}
\newcommand*{\wrt}[0]{w.r.t.\@\xspace}
\newcommand*{\cf}[0]{c.f.\@\xspace}

\makeatletter
\newcommand*{\etc}{%
	\@ifnextchar{.}%
	{etc}%
	{etc.\@\xspace}%
}
\makeatother
%%%




%%%
% Util Macros

% Curly braces {}
\DeclarePairedDelimiter\set\{\}
% Angle brackets < >
\DeclarePairedDelimiter\sv\langle\rangle
%Shared values
\newcommand{\svf}[2][]{\sv[#1]{#2}_f\,}
\newcommand{\svp}[1]{\sv{#1}_\text{plain}}
\newcommand{\svs}[1]{\sv{#1}_\text{sparse}}
\newcommand{\svu}[1]{\sv{#1}_\text{user}}
\newcommand{\svi}[1]{\sv{#1}_\text{item}}

%%%

%%%
% TEMP 

% TODO remove for thesis
% just for dummy text
\usepackage{lipsum}

%schoene TODOs
%\usepackage{todonotes}
%\newcommand{\xtodo}[1]{\todo[color=black!7]{#1}\xspace}
%\newcommand{\itodo}[2][]{\todo[inline,color=green!5,#1]{#2}}
%\newcommand{\ctodo}[2][]{\todo[inline,color=red!5,#1]{#2}}
%
%%%